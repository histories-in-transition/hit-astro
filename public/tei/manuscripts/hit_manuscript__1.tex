
    \documentclass{article}
        \usepackage[ngerman]{babel}
        \usepackage[utf8x]{inputenc}
        \setlength\parindent{0pt}
        \usepackage{hyperref} %for href
        \usepackage{geometry}
        \geometry{
        a4paper,
        textwidth=150mm, %check below or in the xslt also the width of table rows
        left=40mm,
        top=40mm,
        }
        \usepackage{marginnote} % use to add the letters (B, A, L etc.) for the subsections within a description that stand on the left of texts. 
        
        \usepackage[strict]{changepage} %package to change the page when required e.g. width of text
        
        \setlength{\parskip}{0,5em} %controls the space between paragraphs or sections
        
        \usepackage[markcase=noupper]{scrlayer-scrpage} %for header
        \ohead{}% clear the outer head
        \cfoot*{\pagemark}% the pagenumber in the center of the foot, also on plain pages
        
        \reversemarginpar %The report and article document type is single sided, so one needs to add reversemarginpar to get the notes appear on the "wrong" i.e. left side of text.
        \setcounter{secnumdepth}{-1} %removes the automatically generated numbering, -1 stands for the level of chapter, 0 for section etc. Unnumbered sections don't get in the ToC. If needed one has to put a * as in: \section*{Cod. 799}
         
         \title{Beschreibung von Wien, ÖNB, Cod. 387}    
         \author{Beschrieben durch : Leon Pürstinger}
           \date{2025}
         
        \begin{document}  
        \maketitle
      
        
        \begin{adjustwidth}{-20pt}{} %creates environment for the shelf mark and title with smaller left margin.
           Pergament, 165 Bl., 324×256 mm
            \subsection{}
            \newline           
        \end{adjustwidth}
        % Section dividers with marginnotes
        \leavevmode \marginnote{B:}Pergament , 165 Bl., 324×256 mm,
                
                        Foliierung: Neuzeitliche Foliierung, auch auf
                                    Versoseiten.,
                
                \textbf{Lagen:} (II-1)\textsuperscript{3} + 5
                                            IV\textsuperscript{42} + 3 IV\textsuperscript{66} +
                                            (V-1)\textsuperscript{74} + 5 V\textsuperscript{I114} + III\textsuperscript{120} + 2 IV\textsuperscript{136} + (VI-2)\textsuperscript{146} +
                                            V\textsuperscript{156} + (VI-3)\textsuperscript{165}
                \\[0.5\baselineskip]
        
           
           \leavevmode \marginnote{S:} % Layout description
           % Layout info would go here from layoutDesc
           
           \leavevmode \marginnote{A:} % Decoration
           % Decoration info from decoDesc
           
           \leavevmode \marginnote{E:}Datierung 1433. Starker Holzdeckel umfasst von einem weißen, gebräunten
                                    Ledereinband mit doppelten Blindlinien in Rautenform
                \\[0.5\baselineskip]
        
           
           \leavevmode \marginnote{G:}Die Handschrift wurde wohl zusammen mit der Zwillingshandschrift
                            Clm 210 zwischen 810 und 821 (nach Arno Borst 830) in Salzburg
                            angefertigt. Während Clm 210 wahrscheinlich noch im 9. Jahrhundert nach
                            Regensburg gebracht wurde, verblieb der Codex 387 bis zum Jahr 1806 in
                            der Salzburger Dombibliothek und kam anschließend in den Besitz der
                            damaligen Hofbibliothek. 
                        \textbf{Entstehung:}
                        
                                Datierung:  809-818 
                                Ort: Salzburg 
                        \textbf{Provenienz:} Domkapitelbibliothek Salzburg
                \\[0.5\baselineskip]
        
           
           \leavevmode \marginnote{Lit.:} % Bibliography
           
                \textbf{Literatur:} \\
                Bischoff, \textit{Die südostdeutschen Schreibschulen und Bibliotheken in der
                            Karolingerzeit, I}. \\
                Hermann, \textit{}. \\
                Unterkircher, \textit{}. \\
                Borst, \textit{}. \\
                
                \\[0.5\baselineskip]
        
           
           % Contents
           \begin{adjustwidth}{-20pt}{}
           \section{Inhaltsverzeichnis}
           \\[0.5\baselineskip]
           \end{adjustwidth}
           
           
                        \textbf{Zusammenfassung:} \\
                        Einordnung der Handschrift nach den grundlegenden Studien von Arno
                            Borst, die eine aktuellere Klassifikation als die gängigen
                            Bibliothekskataloge bieten. Die Handschrift enthält, wie ihre
                            Schwesternhandschrift Clm 210, den Liber calculationis (Sigle k12 nach
                            Borst). Der Liber besteht aus drei Segmenten, von denen zwei im
                            Inhaltsverzeichnis mit insgesamt 111 Kapitelnummern versehen wurden. Der
                            erste Abschnitt enthält in 99 Kapiteln u.a. einen liturgischen
                            Sonnenjahreskalender (E2) und diverse komputistische Tabellen sowie
                            Anleitungen (darunter verschiedene Auszüge aus Bedas Texten, vgl. k6).
                            Der zweite Block behandelt mit 11 Kapiteln astronomische Inhalte. Das
                            dritte Segement besteht aus Bedas "De natura rerurm" (vgl. k5)
                            und einem umfangreichen Mondzykluskalender (E5). In vorliegender
                            Datenbank wird der Liber calculationis als Gesamtwerk, die drei Segmente
                            und ausgewählte inhaltliche Bestandteile angeführt.\\[0.5\baselineskip]
                \leavevmode \marginnote{1r-3r} \textsc{Inhaltsverzeichnis} (Latein) Üb. \textit{In nomine Domini potentis. Hic capitularium libri incipit
                                calculationis.} Inc. \textit{I Adbreviatio chronicae} Expl. \textit{De mensuris in liquidis} \\[0.3\baselineskip]
        \leavevmode \marginnote{3r} \textsc{Bienensegen} (Latein) Inc. \textit{Apis modicula mater matricula} Expl. \textit{Iohannes vos custodiant} \\[0.3\baselineskip]
        \leavevmode \marginnote{3r} \textsc{Althochdeutscher Eintrag} (Althochdeutsch) Inc. \textit{nuscelihiu} Bemerkung: Althochdeutsches Wort im Stile eines Titels über einem
                                lateinischen Bienensegen ohne erkennbaren Textbezug. \\[0.3\baselineskip]
        \leavevmode \marginnote{1r, 3r} \textsc{Komputistische Anweisungen} (Latein) Bemerkung: Nachträge aus dem 10. Jh. \\[0.3\baselineskip]
        \leavevmode \marginnote{3v} \textsc{Ave maris stella} (Latein) Inc. \textit{Ave maris stella Dei mater alma} Expl. \textit{Vitam praesta puram} \\[0.3\baselineskip]
        \leavevmode \marginnote{4r-7r} \textsc{Abbreviatio chronicae a mundo condito usque
                                ad Carolum Magnum} (Latein) Üb. \textit{I Ad breviatio chronicae} Inc. \textit{Adam cum esset} Expl. \textit{incarnationis Domini IIII DCCLXXXIII} \\[0.3\baselineskip]
        \leavevmode \marginnote{7r, 9v, 10r, 10v, 11r, 15r, 15v} \textsc{Historiographische und komputistische
                                Zusätze} (Latein) Bemerkung: 164v auf Baldo überprüfen. \\[0.3\baselineskip]
        \leavevmode \marginnote{8v-15v} \textsc{Martyrologium excarpsatum cum
                                alphabetis} (Latein) Üb. \textit{V Martyrologium excarpsatum cum alphabetis} Bemerkung: Es handelt sich um einen Sonnenjahreskalender, der mit einem
                                Salzburger Kurzmartyrolog kombiniert wurde. \\[0.3\baselineskip]
        \leavevmode \marginnote{17r-58r} \textsc{Notitiae chronologicae (Ostertafel)} (Latein) Üb. \textit{VII Argumentum ad inveniendam qua XIIII luna paschalis annis
                                singulis feria occurrit cum adscriptis regularibus sive
                                concurrentibus} \\[0.3\baselineskip]
        \leavevmode \marginnote{30v-57v} \textsc{Annales Salisburgenses} (Latein) Inc. \textit{Iustinus minor regnavit} Bemerkung: 
                                graphisch divers zu...
                                Am Rande einer Ostertafel eingefügte annalistische Notizen.
                             \\[0.3\baselineskip]
        \leavevmode \marginnote{1r-165v} \textsc{Liber calculationis} (Latein) \\[0.3\baselineskip]
        \leavevmode \marginnote{115r-130v} \textsc{Astronomische Schriften} (Latein) Üb. \textit{Excerptum de astrologia} Bemerkung: 11 Kapitel \\[0.3\baselineskip]
        \leavevmode \marginnote{130v-146r} \textsc{Beda, Heiliger: De natura rerum} (Latein) Üb. \textit{Versus Bedae presbyteri} Inc. \textit{Incipit Liber} Expl. \textit{a meridiae usque ad occidentem extenditur.} \\[0.3\baselineskip]
        \leavevmode \marginnote{147v-165r} \textsc{Mondkalender} (Latein) Üb. \textit{Cyclus hic est lunaris qualiter luna in circulo decennovali
                                singulis anni vel mensibus sive diebus currit} \\[0.3\baselineskip]
        \leavevmode \marginnote{4r-114v} \textsc{Komputistische Schriften} (Latein) Bemerkung: 99 Kapitel, die sich mit diversen Anleitungen, Tabellen und
                                Rechenbeispielen dem Komputus widmen. \\[0.3\baselineskip]
        
           
           \end{document}
   