
    \documentclass{article}
        \usepackage[ngerman]{babel}
        \usepackage[utf8x]{inputenc}
        \setlength\parindent{0pt}
        \usepackage{hyperref} %for href
        \usepackage{geometry}
        \geometry{
        a4paper,
        textwidth=150mm, %check below or in the xslt also the width of table rows
        left=40mm,
        top=40mm,
        }
        \usepackage{marginnote} % use to add the letters (B, A, L etc.) for the subsections within a description that stand on the left of texts. 
        
        \usepackage[strict]{changepage} %package to change the page when required e.g. width of text
        
        \setlength{\parskip}{0,5em} %controls the space between paragraphs or sections
        
        \usepackage[markcase=noupper]{scrlayer-scrpage} %for header
        \ohead{}% clear the outer head
        \cfoot*{\pagemark}% the pagenumber in the center of the foot, also on plain pages
        
        \reversemarginpar %The report and article document type is single sided, so one needs to add reversemarginpar to get the notes appear on the "wrong" i.e. left side of text.
        \setcounter{secnumdepth}{-1} %removes the automatically generated numbering, -1 stands for the level of chapter, 0 for section etc. Unnumbered sections don't get in the ToC. If needed one has to put a * as in: \section*{Cod. 799}
         
         \title{Beschreibung von München, BSB, Clm 6299}    
         \author{Beschrieben durch : Ksenia Borisova}
           \date{2025}
         
        \begin{document}  
        \maketitle
      
        
        \begin{adjustwidth}{-20pt}{} %creates environment for the shelf mark and title with smaller left margin.
           Pergament, 
            \subsection{}
            \newline           
        \end{adjustwidth}
        % Section dividers with marginnotes
        \leavevmode \marginnote{B:}Pergament , , × mm,
                
                \textbf{Lagen:} 
                \\[0.5\baselineskip]
        
           
           \leavevmode \marginnote{S:} % Layout description
           % Layout info would go here from layoutDesc
           
           \leavevmode \marginnote{A:} % Decoration
           % Decoration info from decoDesc
           
           \leavevmode \marginnote{E:}
           
           \leavevmode \marginnote{G:}
                        \textbf{Entstehung:}
                        
                                Ort: Freising 
                        \textbf{Provenienz:} 
                \\[0.5\baselineskip]
        
           
           \leavevmode \marginnote{Lit.:} % Bibliography
           
           
           % Contents
           \begin{adjustwidth}{-20pt}{}
           \section{Inhaltsverzeichnis}
           \\[0.5\baselineskip]
           \end{adjustwidth}
           
           
                        \textbf{Zusammenfassung:} \\
                        Hieronymus\\[0.5\baselineskip]
                
           
           \end{document}
   