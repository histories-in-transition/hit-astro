
                \documentclass{article}
                \usepackage[ngerman]{babel}
                \usepackage[utf8x]{inputenc}
                \setlength\parindent{0pt}
                \usepackage{hyperref}
                \usepackage{geometry}
                \geometry{
                a4paper,
                textwidth=150mm,
                left=40mm,
                top=40mm,
                }
                \usepackage{marginnote}
                \usepackage[strict]{changepage}
                \setlength{\parskip}{0.5em}
                \usepackage[markcase=noupper]{scrlayer-scrpage}
                \ohead{}
                \cfoot*{\pagemark}
                \reversemarginpar
                \setcounter{secnumdepth}{-1}
                
                \title{Beschreibung von München, BSB, Clm. 6382}
                \author{Ksenia Borisova}
                \date{2025}
                
                \begin{document}
                \maketitle
                
                % Shelf mark and basic info
                \begin{adjustwidth}{-20pt}{}
                \textbf{Sammelbuch}
                
                \smallskip
                 Pergament, 172 Bl. 240×175 mm. Freising, Reims 8 Jh.,
                            drittes Viertel, 9 Jh.; 9 Jh., zweite Hälfte; 8 Jh., Mitte 
                \newline
                \end{adjustwidth}
                
                
                
                        \leavevmode \marginnote{B:}Pergament Lagenformel: (IV + 1)\textsuperscript{9}+ 4 IV\textsuperscript{41}+ I\textsuperscript{43}; 15 IV\textsuperscript{163}+ (V - 1)\textsuperscript{172}
                                
                        
                        \leavevmode \marginnote{S:} 
                        
                        \leavevmode \marginnote{E:}Heller spätgotischer Ledereinband mit Streicheisenlinien.
                                    Schließe defekt, Kettenöse abgerissen. Rücken erneuert. Am
                                    Vorderdeckel je ein Papierschild (15.Jh.) mit Titel, dieser
                                    durch Tintenfraß stark beschädigt, und Nummer M 18.Datierung 1400-1499. \\
                
                \\[0.5\baselineskip]
        
                        
                        \leavevmode \marginnote{G:}Der jüngere erste Teil zeigt auch hinsichtlich der Überlieferung
                            zweimal Beziehung zu Reimser Bibliotheken, bei einem weiteren, in dieser
                            Form seltenen Text sogar eine enge Verwandtschaft zu der im nahen Laon.
                            (Glauche) 
                        \par\textbf{Entstehung:}
                         8 Jh., drittes Viertel Freising  Reims 
                        \par\textbf{Provenienz:} Dombibliothek Freising
                \\[0.5\baselineskip]
        
                        
                        \leavevmode \marginnote{Lit.:}Glauche G., \textit{}. \\
                Bischoff, \textit{Die südostdeutschen Schreibschulen und Bibliotheken in der
                            Karolingerzeit, I}. \\
                
                \\[0.5\baselineskip]
        
                                \begin{adjustwidth}{-20pt}{}
                                \section{Kodikologische Einheit 1 (1r-43v)}
                                \\[0.5\baselineskip]
                                \end{adjustwidth}
                                
                                
                                
                                        \leavevmode \marginnote{A:}
                                        
                       \par Initiale sind rot gefärbt 
                \\[0.5\baselineskip]
        
                                        \leavevmode \marginnote{G:}
                                        
                        \par\textbf{Entstehung:}
                        9 Jh., zweite
                                    HälfteReims
                        \par\textbf{Provenienz:} 
                \\[0.5\baselineskip]
        
                                
                                
                                \begin{adjustwidth}{-20pt}{}
                                \subsection{Inhaltsverzeichnis}
                                \\[0.5\baselineskip]
                                \end{adjustwidth}
                                
                                \leavevmode \marginnote{1r-7r} \textsc{Vita sanctae Euphrosinae} (Latein) Üb. \textit{Incipit vita sanctae Evfrosinę virginis} Inc. \textit{Fuit uir in Alexandria nomine Pafnutius honorabilis
                                    omnibus…} Expl. \textit{… usque in pręsentem diem, glorificantes deum patrem … Cui
                                    est honor…} \\[0.3\baselineskip]
        \leavevmode \marginnote{7r-9v} \textsc{Isidor, Sevilla: Synonyma} (Latein) Üb. \textit{Incipit libellus sinonima} Inc. \textit{Venit nuper ad manus meas…} Expl. \textit{quod ęternum est, sua(lies: supra) modum est, pondus
                                    excellens gloriae‖} Bemerkung: lib. I (fragmentum) \\[0.3\baselineskip]
        \leavevmode \marginnote{10r-20v} \textsc{Ps.-Gregor I: Concordia testimoniorum} (Latein) Üb. \textit{Incipit concordia testimoniorum sancti Gregorii papae urbis
                                    Romae} Inc. \textit{Paulus seruus … Non dico uos‹ seruos…} Expl. \textit{aliis gentibus ueram penitentiam anteacta non
                                    inputarentur.} \\[0.3\baselineskip]
        \leavevmode \marginnote{20v-23v} \textsc{Ps.- Augustinus: De unitate sanctae trinitatis} (Latein) Üb. \textit{Incipit tractatus sancti Agustini episcopi a semetipso ad
                                    semetipsum feliciter} Inc. \textit{Cum me peruigil cura fecisset exsomnem his me questionibus
                                    interrogaui… –… ut quod homines carnalibus caecati curis uidere
                                    non possunt, saltim fidei conpendio nanciscantur. – Contuli ut
                                    potui cum omni solertia. Qui legis ora pro me.} Expl. \textit{Contuli ut potui cum omni solertia. Qui legis ora pro
                                    me.} \\[0.3\baselineskip]
        \leavevmode \marginnote{23v-41r} \textsc{Isidor, Sevilla: Liber interrogatorius de differentiis
                                    rerum et etymologiis secundum Isidorum} (Latein) Üb. \textit{Interrogatio} Inc. \textit{Inter deum et dominum quid interest.} Expl. \textit{Quod dona propriae(proprie!) diuina dicuntur … Hostiae …
                                    antequam ad hostes pergerent, unde ab hostando(!) dictae.
                                    Victimae uero sacrificia‖} \\[0.3\baselineskip]
        \leavevmode \marginnote{41v-42v} \textsc{Breviarium apostolorum} (Latein) Üb. \textit{Incipit breuiarium apostolorum et nomina(lies: ex nomine)
                                    vel locis ubi predicauerunt, orti vel obiti sunt} Inc. \textit{Symon qui interpretatur obędiens, Petrus agnoscens, filius
                                    Johannis, frater Andreę…} Expl. \textit{Mathias … electus sorte et solus sine cognamento(!); cui
                                    datur euangelium(evangelii!) predicatio in Judea} Bemerkung: Nachtrag \\[0.3\baselineskip]
        \leavevmode \marginnote{42v} \textsc{Esra: Revelatio Esdrae de qualitatibus
                                    anni} (Latein) Üb. \textit{Incipit supputatio Esdrae} Inc. \textit{Kalendae Januarii si fuerint dominico die hiems bona erit
                                    et uernus(= ver) uentuosus} Expl. \textit{et auguria ad domus ignis periculum pacientur} \\[0.3\baselineskip]
        \leavevmode \marginnote{42v} \textsc{Anonymus: De temporibus anni} (Latein) Inc. \textit{Vernus exoritur VIII. Kal. Martii permanens diebus
                                    XCI…} \\[0.3\baselineskip]
        \leavevmode \marginnote{42v-43r} \textsc{De somnis} (Latein) Üb. \textit{Incipit de somni (aus somno korrigiert) veris et mendosis
                                    quidam incipiunt in aetatibus lunae exploratis} Inc. \textit{Luna prima quicquid uideris in gaudium conuertitur nec in
                                    malum…} Expl. \textit{Luna intra triduum huius ętatis lunaris somnium peritia
                                    est(im Sinn von expertum est?).} \\[0.3\baselineskip]
        \leavevmode \marginnote{43v} \textsc{Capitulare evangeliorum} (Latein) \\[0.3\baselineskip]
        
                                \begin{adjustwidth}{-20pt}{}
                                \section{Kodikologische Einheit 2 (44r-172v)}
                                \\[0.5\baselineskip]
                                \end{adjustwidth}
                                
                                
                                
                                        \leavevmode \marginnote{A:}
                                        
                       \par Einfache Flechtbandinitiale am Anfang der Kapiteln und
                                    Textstücke. Die feinsten Beispiele stellen die P (105v), Q
                                    (118v), A (135v), Q (146v) dar. Die grössten und feinsten, doch
                                    generell ziemlich grob gezeichneten Initiale kommen auf den
                                    Seiten 45r und 75r vor. 
                \\[0.5\baselineskip]
        
                                        \leavevmode \marginnote{G:}
                                        
                        \par\textbf{Entstehung:}
                        8 Jh., MitteFreising
                        \par\textbf{Provenienz:} 
                \\[0.5\baselineskip]
        
                                
                                
                                \begin{adjustwidth}{-20pt}{}
                                \subsection{Inhaltsverzeichnis}
                                \\[0.5\baselineskip]
                                \end{adjustwidth}
                                
                                \leavevmode \marginnote{44r} \textsc{Erchanbertus, Frisingensis: Epistola de reliquiis sancti
                                    Bartholomaei apostoli nuper in Baioariam allatis} (Latein) Üb. \textit{Erchanbertus dei gratia donante humilis episcopus omnibus
                                    fratribus Christo militantibus domino salutem} Inc. \textit{Denique cognoscat beniuolentia uestra quod quidam uir
                                    nomine Felix …} Expl. \textit{ut istam noticiam aliis inantea non neglegatis
                                    transmittere.} \\[0.3\baselineskip]
        \leavevmode \marginnote{44r-44v} \textsc{Gregor I: Moralia in Job} (Latein) Üb. \textit{Liber moralium super Job} \\[0.3\baselineskip]
        \leavevmode \marginnote{45ra-172rb} \textsc{Gregor I: Moralia in Job} (Latein) Üb. \textit{Incipit moralivm liber tricisimus II(!). Lege
                                    feliciter} Inc. \textit{Sancti uiri quo (aus quod korrigiert) apud deum…} Expl. \textit{Deo domino donante atqve avxiliante moraliorum(!) libri
                                    nvmero XXXV expliciunt.} \\[0.3\baselineskip]
        
                
                \end{document}
        